\documentclass[12pt]{article} %{{{
\usepackage[margin=1in]{geometry}

% Figures
\usepackage{graphicx}
\graphicspath{{../../figs/}}

% Math
\usepackage{amsmath}
\usepackage{amssymb}
\DeclareMathOperator*{\argmin}{\arg\!\min}
\DeclareMathOperator*{\argmax}{\arg\!\max}

% abbreviations
\def\eg{e.g.,~}
\def\ie{i.e.,~}
\def\cf{cf.\ }
\def\viz{viz.\ }
\def\vs{vs.\ }

% Refs
\usepackage{biblatex}
\addbibresource{main.bib}

\usepackage{url}

\newcommand{\secref}[1]{Section~\ref{sec:#1}}
\newcommand{\figref}[1]{Fig.~\ref{fig:#1}}
\newcommand{\tabref}[1]{Table~\ref{tab:#1}}
%\newcommand{\eqnref}[1]{\eqref{eq:#1}}
%\newcommand{\thmref}[1]{Theorem~\ref{#1}}
%\newcommand{\prgref}[1]{Program~\ref{#1}}
%\newcommand{\algref}[1]{Algorithm~\ref{#1}}
%\newcommand{\clmref}[1]{Claim~\ref{#1}}
%\newcommand{\lemref}[1]{Lemma~\ref{#1}}
%\newcommand{\ptyref}[1]{Property~\ref{#1}}

% for quick author comments 
\usepackage[usenames,dvipsnames,svgnames,table]{xcolor}
\definecolor{light-gray}{gray}{0.8}
\def\del#1{ {\color{light-gray}{#1}} }
\def\yy#1{\footnote{\color{red}\textbf{yy: #1}} }

%}}}

\begin{document} %{{{

\title{An awesome paper} %{{{
\date{\today}
\maketitle %}}}

\section{Introduction}\label{sec:introduction} %{{{


%}}}

\section{Results}\label{sec:results} %{{{ 
\subsection{Link prediction algorithms}

%}}}
\textbf{Graph Covolutional Network} The Graph Convolutional Network (GCN) is a convolutional neural network variant specifically designed for analyzing graph-structured data.
In the GCN framework, the convolutional layer performs an aggregation process wherein it combines information from the neighboring nodes of each node in the graph.
This aggregation is achieved by computing a weighted sum of the feature vectors associated wiht the neighboring nodes, as well as the feature vector of the central node itself.
The feature vectors for the graph nodes were derived using the Laplacian Eignenmap embedding. Our implemenation includes two consecutive convolutianal layers.


\section{Methods}\label{sec:methods} %{{{

%}}}

\printbibliography{}
    
\end{document} %}}}
